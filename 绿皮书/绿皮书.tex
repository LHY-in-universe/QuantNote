\documentclass[12pt]{book}

% 中文支持和基础设置
\usepackage[UTF8]{ctex}
\usepackage[a4paper, scale=0.7]{geometry}
\usepackage{mathrsfs}
\usepackage{amssymb}
\usepackage{amsmath}
% 超链接设置
\usepackage{hyperref}
\hypersetup{
    colorlinks=true,
    linkcolor=black,
    filecolor=magenta,
    urlcolor=cyan,
    pdftitle={绿皮书}
}
% 页面布局和格式设置
\usepackage{setspace}
\usepackage{tabularx} 
\onehalfspacing
\setcounter{tocdepth}{2}
\setcounter{secnumdepth}{2}
% 表格优化
\usepackage{booktabs}
\usepackage{array}
% 文档信息
\title{绿皮书 \\ A Practical Guide to Quantitative Finance Interviews}
\author{LHY}
\date{\today}
\begin{document}

\maketitle

\frontmatter % 前言部分(罗马数字页码)
\tableofcontents

\mainmatter % 正文部分(阿拉伯数字页码)

\chapter{General Principles 一般原则}
\begin{itemize}
  \item Build a broad knowledge base
  \item Practice your interview skills
  \item Listen carefully
  \item Speak your mind
  \item Make reasonable assumptions
\end{itemize}

\chapter{Brain Teasers 脑筋急转弯}
\section{Problem Simplification 问题简化}
\subsection{Screwy pirates 疯狂的海盗}
\noindent \textit{Question:}

五个海盗有100个金币, 他们采用以下方式分配:
最年长的海盗提出分配策略, 所有人进行投票, 如果超过50\%的海盗赞同, 通过, 反之最年长的海盗喂鲨鱼. 然后次年长的海盗开始. 假设所有海盗都是完美理性:存活为主, 尽量获得更多金币, 如果两种策略差不多, 船上海盗越少越好. 

\noindent \textit{Solution:} 

考虑两个海盗的简单情况, 海盗代号从1到5, 1大5小. 

对于只有4和5的情况, 无论4提出什么策略都会通过, 所以5会避免出现此种情况. 

对于3、4和5, 3知道如果5在这种策略下一无所获的话, 3就会喂鲨鱼, 所以3给自己99个金币, 给5一个, 这会保障3的策略通过. 在这种情况下, 4一无所获, 所以他要避免这种情况. 

对于2、3、4和5, 2给自己99个, 给4一个, 会保证2的策略通过. 3和5一无所获, 所以会避免这种情况. 

对于1、2、3、4和5, 1给自己98个, 3和5各一个, 1的策略通过. 这也是实际会采取的策略. 

\subsection{Tiger and sheep 老虎和羊}
\noindent \textit{Question:}

一百只老虎和一只羊被放在一个只有草的神奇小岛上. 老虎可以吃草, 但它们更愿意吃羊. 假设A. 每次只能有一只老虎吃一只羊, 而这只老虎吃完羊后自己也会变成一只羊. B. 所有的老虎都很聪明, 而且非常理性, 它们都想生存下去. 那么羊会被吃掉吗?

\noindent \textit{Solution:} 

两只老虎时不会, 三只老虎时会, 四只老虎时不会. 以此类推.

\section{Logic Reasoning 逻辑推理}
\subsection{River crossing 过河问题}
\noindent \textit{Question:}

四个人, A、B、C和D需要过河. 唯一的过河方式是通过一座旧桥, 最多只能容纳两人同时过桥. 由于天黑, 他们不能没有火炬过桥, 而他们只有一个火炬. 所以每对人只能以较慢的人的速度行走. 他们需要尽快地将所有人送到对岸. A是最慢的, 需要10分钟过桥;B需要5分钟;C需要2分钟;D需要1分钟. 那么将所有人送到对岸所需的最短时间是多少?

\noindent \textit{Solution:} 

关键是要认识到, 10 分钟的人应该和 5 分钟的人一起走, 这不应该发生在第一次穿越时, 否则其中一人就必须返回. 因此, C 和 D 应先过河(2 分钟);然后让 D 返回(分钟);A 和 B 过河(10 分钟);让 C 返回(2 分钟);C 和 D 再次过河(2 分钟). 

\subsection{Horse race 赛马}
\noindent \textit{Question:}

这里有25匹马, 每匹马以恒定的速度跑步, 且每匹马的速度都不同于其他马. 由于跑道只有5条道, 每场比赛最多只能有5匹马. 如果你需要找到3匹最快的马, 需要举行的最少比赛次数是多少?

\noindent \textit{Solution:} 

首先举行5场比赛, 得出每场比赛的前三名. 第一比赛, 得出前三名. 第一的第二第三, 第二的第三, 第三进行比赛, 得出前两名.

\section{Thinking Out of the Box 跳出去思考}
\subsection{Box packing 盒子包装}
\noindent \textit{Question:}

把53块 $1\times 1\times 4$ 的砖放进 $6\times 6 \times 6$ 的盒子.

\noindent \textit{Solution:} 

思考 $6\times 6 \times 6$ 分成 27个 $2\times 2 \times 2$ 的小盒子, 14个涂成黑色, 13个涂成白色, 交替涂. 一黑一白最多可以放4个砖, 所以最多可以放 $13\times 4 =52$ 个砖.

\subsection{Calendar cubes 日历方块}
\noindent \textit{Question:}

两个定制骰子, 印上0-9数字, 来显示每个月的日期, 应该怎么安排?

\noindent \textit{Solution:} 

\begin{itemize}
  \item \textbf{第一个}: 0 1 2 3 4 5
  \item \textbf{第二个}: 0 1 2 6 7 8
\end{itemize}

\subsection{Door to offer 幸运门}
\noindent \textit{Question:}

有两扇门, 一扇幸运一扇不幸. 门前有守卫, 一个讲真话, 一个说假话. 只能问一个守卫一个是或者否的问题, 怎么知道幸运门?

\noindent \textit{Solution:} 

问一个守卫“对面那个守卫会告诉我这个门是幸运门吗”.

\subsection{Message delivery 信件传输}
\noindent \textit{Question:}

你需要使用一个盒子给同事传信, 你们各有一把锁, 锁不一样, 只有本人可以打开, 没有上锁的盒子里面的东西会被偷走. 怎么给同事信件?

\noindent \textit{Solution:} 

你先上锁, 给同事后同事上锁, 寄回来你开锁再给同事, 同事开锁.

\subsection{Last ball 最后的球}
\noindent \textit{Question:}

包中有20个蓝球和14个红球, 不放回的拿两个球. 如果同色, 放一个蓝球, 异色, 放一个红球, 你有无限的球. 包中最后的球是什么颜色?

\noindent \textit{Solution:} 

\begin{itemize}
  \item 拿出两个蓝球: 蓝球-1
  \item 拿出两个红球: 红球-2, 蓝球+1
  \item 拿出异色: 蓝球-1
\end{itemize}

如果是14个红球, 红球一定成对拿走, 最后一个是蓝球.
如果是13个红球, 最后一个是蓝球.

\subsection{Quant salary 薪水问题}
\noindent \textit{Question:}

如何在不知道其他人工资的情况下计算平均工资?

\noindent \textit{Solution:} 

第一个人的工资加随机数, 如何传给其他人, 最后第一个人减去随机数得到平均工资.

\section{Application of Symmetry 对称性的使用}
\subsection{Coin piles 硬币堆}
\noindent \textit{Question:}

在一个黑暗的房间里面, 有1000枚硬币, 980枚朝上, 20枚朝下. 你可以无数次反转硬币, 可以把硬币分成两堆, 朝下的个数一样吗?

\noindent \textit{Solution:} 

随机找到20个分成一堆, 全部反转, 即可达成目标.

\subsection{Mislabeled bags 错误标签的书包}
\noindent \textit{Question:}

有三个书包, 一个全是苹果, 一个全是橘子, 一个是苹果和橘子的混合, 但是标签全部错误. 最少拿多少个水果, 可以分辨出来.

\noindent \textit{Solution:} 

因为标签全部错误, 所以只需要看混合书包. 混合书包一定的纯的, 拿出一个水果, 就可以判断全部.

\subsection{Wise men 智者}
\noindent \textit{Question:}

国王抓了50个智者, 他有一个反着的杯子, 每分钟他可以随机叫一个智者来反转或不动杯子. 当有人正确地说他已经叫了全部智者, 那么所有智者得救. 所有智者只能交流一次. 有什么策略可以使所有人得救?

\noindent \textit{Solution:} 

选出一个传话者, 他每次见到正的杯子会倒过来, 剩下的人第一次看见倒着的杯子要正过来. 传话者进行计数, 49次时即可.

\section{Series Summation 级数相加}
$$\sum_{n=1}^N n =\frac{N(N+1)}{2} $$
$$\sum_{n=1}^N n^2=\frac{N(N+1)(2N+1)}{6}=\frac{N^3}{3}+\frac{N^2}{2}+\frac{N}{6}$$

\subsection{Missing integers 丢失的整数}
\noindent \textit{Question:}

在1-100的范围里面, 有98个不一样的数字, 怎么找到两个失去的数字?

\noindent \textit{Solution:} 

计算98个数字的和和平方和, 与1-100的结果做差, 解方程组可得.

\subsection{Counterfeit coins 找出假币}
\noindent \textit{Question:}

10个袋子, 每个袋子有100个硬币, 所有硬币重10克, 只有一个袋子有假币, 假币重9或者11克. 怎么使用一个显示精确重量的电子秤一次称出那个袋子有假币?

\noindent \textit{Solution:} 

每一个袋子各自取不同数量的硬币, 一起称量. 看理论值与实际值的差, 即可猜出哪个袋子有假币.

\subsection{Glass balls 玻璃球}
\noindent \textit{Question:}

你在100层楼上, 有两个玻璃球. 你可以往下扔球, 超过X层时, 球会破. 考虑最坏的情况, 为了得到X, 最少要扔多少次球?

\noindent \textit{Solution:} 

假设在N层扔球, 球破了, 最坏要N次才可以得出. 球没有破, 尝试在N+N-1层扔球, 球破, 最坏要N次确定. 球没有破, 尝试N+N-1+N-2层, 以此类推最多层数为N(N+1)/2, 对于100, N=14.

\section{The Pigeon Hole Principle 鸽巢原理}
如果鸽巢数量比鸽子少, 那么在将所有鸽子放到鸽巢后, 一定有至少一个鸽巢容纳多只鸽子.

\subsection{Matching socks 分袜子}
\noindent \textit{Question:}

你抽屉里面有2只红外子, 20只黄袜子, 31只蓝袜子. 你要随机在抽屉拿多少只袜子, 才可以保证有成对的袜子?

\noindent \textit{Solution:} 

一共3个颜色, 4只就可以.

\subsection{Have we met before 拉姆齐理论}
\noindent \textit{Question:}

6个人的聚会中, 一定有三个人互相不认识或者互相认识.

\noindent \textit{Solution:} 

对A来说, 至少3人认识或者不认识A. 这三个人中或者最少两个人认识, 或者都不认识.

\subsection{Ants on a square}
\noindent \textit{Question:}

有51个蚂蚁在长度为1的广场上, 是否可以拿一个直径1/7的圆盘盖住至少三个蚂蚁?

\noindent \textit{Solution:} 

将广场分成25部分, 其中至少有一个区域最少有三只蚂蚁.

\section{Modular Arithmetic 模运算}
\subsection{Prisoner problem 囚犯问题}
\noindent \textit{Question:}

明天, 一百名囚犯将获得自由的机会. 他们都被告知, 每人将被分配戴红色或蓝色的帽子. 每名囚犯都能看到其他人的帽子, 但看不到自己的帽子. 帽子的颜色是随机分配的, 一旦帽子被戴到头上, 他们就不能以任何形式与他人交流, 否则他们将被立即处决. 囚犯将被随机叫出, 叫出的囚犯将猜测自己的帽子的颜色. 每名囚犯都会公开宣布自己的帽子的颜色, 以便其他人都能听到. 如果囚犯正确猜测了自己的帽子的颜色, 他将被立即释放;否则他将被处决. 他们被给予一夜来制定策略, 以拯救尽可能多的囚犯. 他们可以采取什么样的策略, 并且可以保证拯救多少囚犯?

\noindent \textit{Solution:} 

最少99个. 第一个犯人看红色帽子个数, 奇数就是红色, 反之蓝色. 其他人根据他的回答判断自己的颜色.


\subsection{Division by 9 9的倍数}
\noindent \textit{Question:}

证明任意一个数, 若所有整数位相加是9的倍数, 则该数是9的倍数.

\noindent \textit{Solution:} 

$a=a_n 10^n+a_{n-1} 10^{n-1}+...+a_1 10 +a_0 $, 若 $a_n+...+a_1+a_0=9x$, 则 $a-9x=a_n(10^n-1)+a_{n-1}(10^{n-1}-1)+...+a_1(10-1) \% 9 =0$


\section{Math Induction 数学归纳法}
\subsection{Chocolate bar problem 巧克力棒问题 }
\noindent \textit{Question:}

一个巧克力有 $6\times 8$ 样子, 可以把巧克力分成两份 $6\times 3, 6\times 5$. 多少次可以把巧克力分成48份?

\noindent \textit{Solution:} 

每次断裂都可以使巧克力数量+1, 所以有48-1=47次.

\section{Proof by Contradiction 反证法}
\subsection{Irrational number 无理数}
\noindent \textit{Question:}

有理数可以使用分数表示, 证明$\sqrt{2}$是无理数.

 \noindent \textit{Solution:} 
 
 假设 $\sqrt{2}=\frac{m}{n}$, 其中 $mn $不可约分. 可得$2n^2=m^2$, m是偶数, $m=2x, m^2=4x^2, n^2=2x^2$, 得出n是偶数, mn可以约分.
 
  
\chapter{Probability Theory 概率论}
\section{Basic Probability Definitions and Set Operations 基本概念}

\textbf{Outcome} $(\omega)$: The outcome of an experiment or trial

\textbf{Sample space OR Probability space} $\Omega$: The set of all possible outcomes of an experiment.

\subsection{Coin toss game 抛硬币游戏}
\noindent \textit{Question:}

A有n+1个硬币, B有n个硬币. 抛完所有硬币后, A朝上的数量比B多的概率是?

 \noindent \textit{Solution:} 
 
 考虑A有n个硬币, 这时候有对称性. 一共三种可能, 一方多或者一样多. 可得式子 $2x+y=1$. 考虑A多出的一个硬币, A比B多的概率变成 $x+0.5y=x+0.5(1-2x)=0.5$

\subsection{Drunk passenger 喝醉的乘客}
\noindent \textit{Question:}

有100个人坐飞机, 每个人都有自己的座位. 第一个乘客喝醉了, 随机坐了一个座位. 如果其他乘客发现自己的座位被占, 会随机挑选一个座位. 第一百个乘客正确做到自己座位的概率?

 \noindent \textit{Solution:} 
 
 只考虑第一个座位和第一百个座位, 只有两种可能. 第一个座位在第一百个之前坐, 或在之后坐. 两个事件概率相同. 对于第一种情况, 会成功做到自己座位上面, 概率是0.5.
 
 \subsection{N points on a circle 圆上的点}
\noindent \textit{Question:}


N个点随机分布在圆上, 它们都在一个半圆的概率是多少?

 \noindent \textit{Solution:} 
 
 若选出任意一点为起始点, 则所有点都在半圆的概率是 $\frac{1}{2}^{N-1}$, 因为所有点等价, 所以最终答案是 $N \times \frac{1}{2}^{N-1}$.
 
 \section{Combinatorial Analysis 组合分析}

\textbf{Permutation}: A rearrangement of objects into distinct sequence. $$ \frac{n!}{n_1!n_2!...n_r!}$$

\textbf{Combination}: An unordered collection of objects. 
$$\binom{n}{r}=\frac{n!}{(n-r)!r!}$$

\textbf{Binomial theorem}: $(x+y)^n=\sum_{k=0}^n\binom{n}{k}x^ky^{n-k}$

 \subsection{Hopping rabbit 跳跃的兔子}
\noindent \textit{Question:}

兔子在n层台阶上, 可以一次跳下一层或者两层. 有多少种方式跳到地面?

 \noindent \textit{Solution:} 
 
 跳下一层时, 问题变成n-1层, 跳下两层时, 问题变成n-2层. 所以可以写出 $f(n)=f(n-1)+f(n-2)$. 根据n=1和n=2的情况, 可以得出解是斐波那契数列.
 
  \subsection{Screwy pirates 疯狂的海盗}
\noindent \textit{Question:}

11个海盗把宝藏放在一个保险箱里面, 保险箱有很多锁, 每个人有很多钥匙. 只有在最少6个人拥有相同锁的钥匙时, 保险箱才可以打开. 为了保证随机挑选的6个海盗都可以打开箱子, 需要多少锁, 每个海盗要带多少钥匙?

 \noindent \textit{Solution:} 
 
 需要 $\binom{11}{6}=462$ 把锁, 那么一共 $462\times 6$ 把钥匙, 每个人要带$462\times 6/11=252$ 把钥匙.

  \subsection{Derangement 完全错位排列}
\noindent \textit{Question:}

在排列中没有任何一个元素保持在原来的位置上的概率.

 \noindent \textit{Solution:} 
 
 $$D_n=n!\left(1-\frac{1}{1!}+\frac{1}{2!}-\frac{1}{3!}+\cdots+(-1)^n\frac{1}{n!}\right)$$ 
 $$D_n=\left\lfloor\frac{n!}{e}+\frac{1}{2}\right\rfloor$$

  \subsection{Birthday problem 生日问题}
\noindent \textit{Question:}

一年有365天, 为了让两个同学有相同生日的概率大于0.5, 班里要有多少同学.

 \noindent \textit{Solution:} 
 
 可以转换为班里没有同学有相同生日的概率小于0.5. 即 $\frac{365\times364\times\cdots\times(365-n+1)}{365^n}<1/2$. n=23.
 
   \subsection{Cubic of integer 整数立方}
\noindent \textit{Question:}

x是 $1-10^{12}$ 的一个整数, x的立方的最后两个数字是11的概率是多少?

 \noindent \textit{Solution:} 
 
 x总可以写为 a+10b, a是x的最后一位. 这样 $x^3=(a+10b)^3=a^3+30a^2b+300ab^2+1000b^3$, 最后一位数字只和a有关, 得出a=1. 进而推出b的最后一位是7, x的最后两位是71. 
 
 \section{Conditional Probability and Bayes's formula 条件概率和贝叶斯公式}

\textbf{Conditional probability P(A|B)}: If P(B) > 0, then $P(A|B) = \frac{P(AB)}{P(B)}$
is the fraction
of B outcomes that are also A outcomes.

\textbf{Multiplication Rule}: $$P(E_1E_2\cdots E_n)=P(E_1)P(E_2\mid E_1)P(E_3\mid E_1E_2)\cdots P(E_n\mid E_1\cdots E_{n-1})$$

\textbf{Bayes' Formula}: $$P(F_j\mid E)=\frac{P(E\mid F_j)P(F_j)}{\sum_{i=1}^nP(E\mid F_i)P(F_i)}$$ 

   \subsection{Boys and girls 男孩女孩}
\noindent \textit{Question:}

公司为至少有一个男孩的母亲举行宴会, Jackson有两个孩子, 被邀请了, 那么两个孩子都是男生的概率是?

 \noindent \textit{Solution:} 
 
B代表最少有一个男孩的事件, A代表都是男孩的事件. $P(A\mid B)=\frac{P(A\cap B)}{P(B)}=\frac{P\left(\{(b,b)\}\right)}{P\left(\{(b,b),(b,g),(g,b)\}\right)}=\frac{1/4}{3/4}=\frac{1}{3}$ 

   \subsection{Unfair coin 不公平的硬币}
\noindent \textit{Question:}

你有1000枚硬币, 999正常, 一个不正常有两个正面. 你随机选择一个硬币, 扔了10次都朝上. 那么选出的是不正常的硬币的概率是?

 \noindent \textit{Solution:} 
 
 应用贝叶斯公式, $P(A\mid B)=\frac{P(B\mid A)P(A)}{P(B)}=\frac{P(B\mid A)P(A)}{P(B\mid A)P(A)+P(B\mid A^c)P(A^c)}$. $A$ 是选出错误硬币事件, $A^c$ 是选出正确硬币事件, B是全部朝上事件. 带入可得结果, $P(A\mid B)=\frac{P(B\mid A)P(A)}{P(B\mid A)P(A)+P(B\mid A^c)P(A^c)}=\frac{1/1000\times1}{1/1000\times1+999/1000\times1/1024}\approx0.5.$
 
    \subsection{Von Neumann’s Method 冯·诺伊曼方法}
\noindent \textit{Question:}
 
 对于一个坏硬币, 正反面概率不相等. 有什么办法可以使用这个硬币测0.5概率事件?
 
  \noindent \textit{Solution:} 
  
  连续扔两次, 正反和反正的概率一样, 舍弃正正和反反情况.
 
     \subsection{Dice order 骰子的顺序}
\noindent \textit{Question:}
 
 连着扔三个骰子, 结果严格递增的概率?
 
   \noindent \textit{Solution:} 
   
   概率是三次结果不一样且只按一种顺序. $$\begin{aligned}\mathrm{P}&=P(\text{different numbers in all three throws})\times P(\text{increasing order|3 different numbers})\\&=(1\times\frac{5}{6}\times\frac{4}{6})\times\frac{1}{6}=5/54\end{aligned}$$
 
 \subsection{Monty Hall problem 蒙提霍尔问题}
\noindent \textit{Question:} 
 
 你面前有三扇门, 一扇后面是汽车, 其余是山羊. 你随机选择一扇, 这时主人打开一扇有山羊并且没有被选择的门. 那么你是否应该换门?
 
\noindent \textit{Solution:} 
    
如果最开始选择山羊(2/3), 换门会得到车. 最开始车(1/3), 换门会得到山羊, 不换门是车. 所以应该换门.    
 
 \subsection{Branching Process 分支过程}
\noindent \textit{Question:}  
 
 一个变形虫有四种状态:死亡, 保持不变, 分裂为两个和三个. 后代完全一样. 则该变形虫种群灭绝的概率?
  
\noindent \textit{Solution:} 

假设一只灭绝的概率是 $P(E)$, 那么由$$P(E)=P(E\mid F_1)P(F_1)+P(E\mid F_2)P(F_2)+\cdotp\cdotp\cdotp+P(E\mid F_n)P(F_n)$$ 得到 $$P(E)=1/4\times1+1/4\times P(E)+1/4\times P(E)^2+1/4\times P(E)^3$$ 解得 $P(E)=\sqrt{2}-1\approx0.414$.
 
  \subsection{Candies in a jar 蜜罐中的糖果}
\noindent \textit{Question:}  

在一个罐子中有10个红球, 20个蓝球, 30个绿球. 不放回拿球, 在拿完所有红球后, 至少还有一个蓝球和绿球的概率?

\noindent \textit{Solution:} 

所有球中最后一个球是绿球的概率是 $\frac{30}{60}$, 在所有红球和蓝球中最后一个球是蓝球的概率是 $\frac{20}{30}$. 以此类推, 可得出总概率为 $\frac{30}{60}\times \frac{20}{30}+\frac{20}{60}\times\frac{30}{40}=\frac{7}{12}$.

  \subsection{Coin toss game 扔硬币游戏}
\noindent \textit{Question:}

 A和B扔硬币, A先开始. 如果最后两次先上后下, 游戏结束, 扔下的人获胜. A获胜的概率?
 
 \noindent \textit{Solution:} 
 
 $P(A)=0.5P(A|H)+0.5P(A|T)$ 是A获胜的概率, $1-P(A)$ 是B获胜的概率. A如果第一次下, 游戏等价B先扔. $P(A|T)=1-P(A)$, A第一次上, B有一半概率赢, 有一半概率上. 这时有 $$P(A|H)=0.5\times 0 +0.5\times (1-P(A|H))\Rightarrow P(A|H)=1/3 $$
 带入得 $P(A)=1/2\times 1 /3 +1/2 (1-P(A))\Rightarrow P(A)=4/9 $.
 
   \subsection{Russian roulette series 俄罗斯转轮}
\noindent \textit{Question:}
 
 一把左轮手枪有6个子弹孔, 只有一发子弹, 一个人扣动扳机后另一个人继续. 你要当第一个还是第二个? 如果每次扣动扳机后打乱顺序, 你要当第一个还是第二个?
 
 \noindent \textit{Solution:} 
 
  因为子弹位置随机, 所以两个人死亡概率相等.


假设第一个人死亡概率是 $p$, 第二个人就是 $1-p$. 因为事件都是相互独立的, 那么在第一个人没死后, 他会变成第二个人. $p=1/6+5/6\times (1-p)\Rightarrow p =6/11$.
 
 
   \subsection{Gambler's Ruin Problem 赌徒破产问题}
\noindent \textit{Question:} 

 一个赌徒有 $p$ 概率赢得一元, $1-p$ 概率输掉一元. 最开始有 $i$元, 那么资产达到 $N$ 元的概率?
 
 \noindent \textit{Solution:} 
 
 假设 $P_i$ 是从i到N的概率. 那么$P_0=0, P_N=1$. 
 $$ P_i=p P_{i+1}+qP_{i-1} \Rightarrow  P_{i+1}-P_i=\frac{q}{p}(P_i-P_{i-1})=(\frac{q}{p})^2(P_{i-1}-P_{i-2})=...= (\frac{q}{p})^{i}(P_1-P_0) $$
 
 使用 $P_N$ 取得 $P_1$, 即可得到答案.

   \subsection{Polya urn model 波利亚罐子模型}
\noindent \textit{Question:} 
 
 从一个罐子中拿球, 有红黑两种, 第一次红球, 第二次黑球. 第n次拿到红球的概率是前n-1次拿到的红球比n-1. 那么在第一百次拿到50个红球的概率?
 
 \noindent \textit{Solution:} 
 
 引入记号$(n,k)$, 前n次有k个红球. 可以得到 $P_{3,1}=1/2, P_{3,2}=1/2$. $$\left\{\begin{array}{l}P_{4,1}=P((4,1)\mid(3,1))\times P_{3,1}+P((4,1)\mid(3,2))\times P_{3,2}=\frac{2}{3}\times\frac{1}{2}+0\times\frac{1}{2}=\frac{1}{3}\\P_{4,2}=P((4,2)\mid(3,1))\times P_{3,1}+P((4,2)\mid(3,2))\times P_{3,2}=\frac{1}{3}\times\frac{1}{2}+\frac{1}{3}\times\frac{1}{2}=\frac{1}{3}\\P_{4,3}=P((4,3)\mid(3,1))\times P_{3,1}+P((4,3)\mid(3,2))\times P_{3,2}=0\times\frac{1}{2}+\frac{2}{3}\times\frac{1}{2}=\frac{1}{3}\end{array}\right.$$
 
 使用数学归纳法猜测 $P_{n,k}=\frac{1}{n-1}$. $$\begin{aligned}&P_{n+1,k}=P\left(miss|(n,k)\right)P_{n,k}+P\left(score|(n,k-1)\right)P_{n,k-1}\\&=\left(1-\frac{k}{n}\right)\frac{1}{n-1}+\frac{k-1}{n}\frac{1}{n-1}=\frac{1}{n}\end{aligned}$$
 
    \subsection{Cars on road 路上的车}
\noindent \textit{Question:}
 
 在高速公路上任意20分钟至少看见一辆车的概率是609/625, 那么在任意5分钟看见一辆车的概率是?
 
 \noindent \textit{Solution:} 
 
 将任意一个时间段分成4份, 假设在一个5分钟内看见车的概率是$p$, 那么在四段时间都看不见车的概率为 $(1-p)^4=1-609/625$, 得出 $p=3/5$.
 
 \section{Discrete and Continuous Distributions 离散与连续分布} 
 
\begin{table}[h]
\caption{Discrete Probability Distributions}
\label{tab:distributions}
\begin{tabularx}{\linewidth}{@{} l X c c @{}}  % @{}消除左右边距
\toprule
\textbf{Name} & \textbf{Probability Mass Function (pmf)} & $\mathbf{E[X]}$ & $\mathbf{\text{Var}(X)}$ \\
\midrule
Uniform &
$P(x) = \dfrac{1}{b - a + 1}, \quad x = a, \ldots, b$ &
$\dfrac{b + a}{2}$ &
$\dfrac{(b - a + 1)^2 - 1}{12}$ \\
\addlinespace[2pt]
Binomial &
$P(x) = \dbinom{n}{x} p^x (1 - p)^{n - x}, \quad x = 0, \ldots, n$ &
$n p$ &
$n p (1 - p)$ \\
\addlinespace[2pt]
Poisson &
$P(x) = \dfrac{e^{-\lambda t} (\lambda t)^x}{x!}, \quad x = 0, 1, \ldots$ &
$\lambda t$ &
$\lambda t$ \\
\addlinespace[2pt]
Geometric &
$P(x) = (1 - p)^{x - 1} p, \quad x = 1, 2, \ldots$ &
$\dfrac{1}{p}$ &
$\dfrac{1 - p}{p^2}$ \\
\addlinespace[2pt]
Negative Binomial &
$P(x) = \dbinom{x - 1}{r - 1} p^r (1 - p)^{x - r}, x = r, r+1, \ldots$ &
$\dfrac{r}{p}$ &
$\dfrac{r (1 - p)}{p^2}$ \\
\bottomrule
\end{tabularx}
\end{table}

\begin{table}[htbp]
\centering
\caption{Continuous Probability Distributions}
\label{tab:continuous_dist}
\begin{tabularx}{\linewidth}{@{} l X c c @{}}  % @{}消除左右边距
\toprule
\textbf{Name} & \textbf{Probability Density Function (pdf)} & $\mathbf{E[X]}$ & $\mathbf{Var(X)}$ \\
\midrule
Uniform & 
$\dfrac{1}{b-a},\quad a \leq x \leq b$ & 
$\dfrac{a+b}{2}$ & 
$\dfrac{(b-a)^2}{12}$ \\
\addlinespace[3pt]
Normal & 
$\dfrac{1}{\sqrt{2\pi}\sigma}\exp\left(-\dfrac{(x-\mu)^2}{2\sigma^2}\right),\quad x \in \mathbb{R}$ & 
$\mu$ & 
$\sigma^2$ \\
\addlinespace[3pt]
Exponential & 
$\lambda e^{-\lambda x},\quad x \geq 0$ & 
$\dfrac{1}{\lambda}$ & 
$\dfrac{1}{\lambda^2}$ \\
\addlinespace[3pt]
Gamma & 
$\dfrac{\lambda^\alpha}{\Gamma(\alpha)}x^{\alpha-1}e^{-\lambda x},\quad x \geq 0$ \newline
$\Gamma(\alpha) = \displaystyle\int_0^\infty y^{\alpha-1}e^{-y}\,dy$ & 
$\dfrac{\alpha}{\lambda}$ & 
$\dfrac{\alpha}{\lambda^2}$ \\
\addlinespace[3pt]
Beta & 
$\dfrac{\Gamma(\alpha+\beta)}{\Gamma(\alpha)\Gamma(\beta)}x^{\alpha-1}(1-x)^{\beta-1},0 < x < 1$ & 
$\dfrac{\alpha}{\alpha+\beta}$ & 
$\dfrac{\alpha\beta}{(\alpha+\beta)^2(\alpha+\beta+1)}$ \\
\bottomrule
\end{tabularx}
\end{table}
 
 

 
 
 
 
 
 
 
 
 
 
 
 
 
 
 
 
 
 
 
 
 
 
 
 
 
 
 
 
 
 
 
 
 
 
 
 
 
 
 
 
 
 
 
 
 
 
 
 
 
 
 
 
 
 
 
 
 
 
 
 
 
 
 
 
 
 
 
 
 
 
 
 
 
 
 
 
 
 
 
 
 
 
 
 
 
 
 
 
 
 
 
 
 
 
 
 
 
 
 
 
 
 
 
 
 
 
 
 
 
 
 
 
 
 
 
 
 
 
 
 
 
 
 
 
 
 
 
 
 
 
 
 
 
 
 
 
 
 
 
 
 
 
 
 
 
 

 
 
 
 
 
 
 
 
 
 
 
 
 





\end{document}
